% ARTICLE 2 ----
% This is just here so I know exactly what I'm looking at in Rstudio when messing with stuff.
% Options for packages loaded elsewhere
\PassOptionsToPackage{unicode}{hyperref}
\PassOptionsToPackage{hyphens}{url}
%
\documentclass[
  11pt,
]{article}
\usepackage{lmodern}
\usepackage{amssymb,amsmath}
\usepackage{ifxetex,ifluatex}
\ifnum 0\ifxetex 1\fi\ifluatex 1\fi=0 % if pdftex
  \usepackage[T1]{fontenc}
  \usepackage[utf8]{inputenc}
  \usepackage{textcomp} % provide euro and other symbols
\else % if luatex or xetex
  \usepackage{unicode-math}
  \defaultfontfeatures{Scale=MatchLowercase}
  \defaultfontfeatures[\rmfamily]{Ligatures=TeX,Scale=1}
  \setmainfont[]{cochineal}
\fi
% Use upquote if available, for straight quotes in verbatim environments
\IfFileExists{upquote.sty}{\usepackage{upquote}}{}
\IfFileExists{microtype.sty}{% use microtype if available
  \usepackage[]{microtype}
  \UseMicrotypeSet[protrusion]{basicmath} % disable protrusion for tt fonts
}{}
\makeatletter
\@ifundefined{KOMAClassName}{% if non-KOMA class
  \IfFileExists{parskip.sty}{%
    \usepackage{parskip}
  }{% else
    \setlength{\parindent}{0pt}
    \setlength{\parskip}{6pt plus 2pt minus 1pt}
    }
}{% if KOMA class
  \KOMAoptions{parskip=half}}
\makeatother
\usepackage{xcolor}
\IfFileExists{xurl.sty}{\usepackage{xurl}}{} % add URL line breaks if available
\urlstyle{same} % disable monospaced font for URLs
\usepackage[margin=1in]{geometry}
\setlength{\emergencystretch}{3em} % prevent overfull lines
\providecommand{\tightlist}{%
  \setlength{\itemsep}{0pt}\setlength{\parskip}{0pt}}
\setcounter{secnumdepth}{-\maxdimen} % remove section numbering

\ifluatex
  \usepackage{selnolig}  % disable illegal ligatures
\fi
\usepackage[]{natbib}
\bibliographystyle{apsr}


\title{Aid for Alliances? Buying and Selling Alliances in the Political
Economy of Aid}
\author{true}
\date{August 17, 2022}

% Jesus, okay, everything above this comment is default Pandoc LaTeX template. -----
% ----------------------------------------------------------------------------------
% I think I had assumed beamer and LaTex were somehow different templates.


\usepackage{kantlipsum}

\usepackage{abstract}
\renewcommand{\abstractname}{}    % clear the title
\renewcommand{\absnamepos}{empty} % originally center

\renewenvironment{abstract}
 {{%
    \setlength{\leftmargin}{0mm}
    \setlength{\rightmargin}{\leftmargin}%
  }%
  \relax}
 {\endlist}

\makeatletter
\def\@maketitle{%
  \newpage
%  \null
%  \vskip 2em%
%  \begin{center}%
  \let \footnote \thanks
      {\fontsize{18}{20}\selectfont\raggedright  \setlength{\parindent}{0pt} \@title \par}
    }
%\fi
\makeatother


\title{Aid for Alliances? Buying and Selling Alliances in the Political
Economy of Aid }

\date{}

\usepackage{titlesec}

% 
\titleformat*{\section}{\large\bfseries}
\titleformat*{\subsection}{\normalsize\itshape} % \small\uppercase
\titleformat*{\subsubsection}{\normalsize\itshape}
\titleformat*{\paragraph}{\normalsize\itshape}
\titleformat*{\subparagraph}{\normalsize\itshape}

% add some other packages ----------

% \usepackage{multicol}
% This should regulate where figures float
% See: https://tex.stackexchange.com/questions/2275/keeping-tables-figures-close-to-where-they-are-mentioned
\usepackage[section]{placeins}



\makeatletter
\@ifpackageloaded{hyperref}{}{%
\ifxetex
  \PassOptionsToPackage{hyphens}{url}\usepackage[setpagesize=false, % page size defined by xetex
              unicode=false, % unicode breaks when used with xetex
              xetex]{hyperref}
\else
  \PassOptionsToPackage{hyphens}{url}\usepackage[draft,unicode=true]{hyperref}
\fi
}

\@ifpackageloaded{color}{
    \PassOptionsToPackage{usenames,dvipsnames}{color}
}{%
    \usepackage[usenames,dvipsnames]{color}
}
\makeatother
\hypersetup{breaklinks=true,
            bookmarks=true,
            pdfauthor={Miles D. Williams (Denison University)},
             pdfkeywords = {pandoc, r markdown, knitr},
            pdftitle={Aid for Alliances? Buying and Selling Alliances in
the Political Economy of Aid},
            colorlinks=true,
            citecolor=blue,
            urlcolor=blue,
            linkcolor=magenta,
            pdfborder={0 0 0}}
\urlstyle{same}  % don't use monospace font for urls

% Add an option for endnotes. -----



% This will better treat References as a section when using natbib
% https://tex.stackexchange.com/questions/49962/bibliography-title-fontsize-problem-with-bibtex-and-the-natbib-package
\renewcommand\bibsection{%
   \section*{References}%
   \markboth{\MakeUppercase{\refname}}{\MakeUppercase{\refname}}%
  }%

% set default figure placement to htbp
\makeatletter
\def\fps@figure{htbp}
\makeatother



\usepackage{longtable}
\LTcapwidth=.95\textwidth
\linespread{1.05}
\usepackage{hyperref}

\newtheorem{hypothesis}{Hypothesis}

\usepackage{setspace}

% trick for moving figures to back of document
% really wish we'd knock this shit off with moving tables/figures to back of document
% but, alas...

% 
% Optional code chunks ------
% SOURCE: https://stackoverflow.com/questions/50702942/does-rmarkdown-allow-captions-and-references-for-code-chunks



\begin{document}

% \textsf{\textbf{This is sans-serif bold text.}}
% \textbf{\textsf{This is bold sans-serif text.}}


% \maketitle

{% \usefont{T1}{pnc}{m}{n}
\setlength{\parindent}{0pt}
\thispagestyle{plain}
{%\fontsize{18}{20}\selectfont\raggedright
\maketitle  % title \par

}




{
   \vskip 13.5pt\relax \normalsize\fontsize{11}{12}
   \MakeUppercase{Miles D. Williams}, \small{Denison University}   

}

}








\begin{abstract}

%    \hbox{\vrule height .2pt width 39.14pc}

    \vskip 8.5pt % \small

\noindent \small{Lorem ipsum dolor sit amet, consectetur adipiscing
elit. Donec sit amet libero justo. Pellentesque eget nibh ex. Aliquam
tincidunt egestas lectus id ullamcorper. Proin tellus orci, posuere sed
cursus at, bibendum ac odio. Nam consequat non ante eget aliquam. Nulla
facilisis tincidunt elit. Nunc hendrerit pellentesque quam, eu imperdiet
ipsum porttitor ut. Interdum et malesuada fames ac ante ipsum primis in
faucibus. Suspendisse potenti. Duis vitae nibh mauris. Duis nec sem sit
amet ante dictum mattis. Suspendisse diam velit, maximus eget commodo
at, faucibus et nisi. Ut a pellentesque eros, sit amet suscipit eros.
Nunc tincidunt quis risus suscipit vestibulum. Quisque eu fringilla
massa.}


\vskip 8.5pt \noindent \emph{Keywords}: pandoc, r markdown, knitr \par

%    \hbox{\vrule height .2pt width 39.14pc}



\end{abstract}


\vskip -8.5pt


 % removetitleabstract


\setlength{\parindent}{16pt}
\setlength{\parskip}{0pt}

% We'll put doublespacing here
\doublespacing
% Remember to cut it out later before bib
This study is important for three reasons. First, it contributes to a
growing literature that recognizes linkages between issues in
international political economy and international security that
historically have been studied in isolation. In particular, this study
demonstrates how both foreign aid and military alliances (highly
important and regularly studied variables in these respective fields)
are deeply intertwined.

Second, this study contributes to our understanding of the role that
alliances play in interactions between powerful industrialized countries
and comparatively weak developing countries. Specifically, this study
underlines how alliances operate as a form of exchange but with
differing implications depending on the content alliance promises. Of
the two most common kinds of alliances that exist between industrialized
countries and developing countries---nonaggression pacts and defensive
pacts---the former reflects a concession made by weaker states to strong
ones while the latter reflects a concession made by stronger states to
weak ones. The demonstration of this difference in the direction of
exchange implied by alliance provisions adds to a growing literature
that continues to probe the content of alliances and how different
security pacts lead to different kinds of behavior by signatories.

Third, and finally, this study joins more recent contributions to the
aid literature that have begun to exploit more granular data on aid
sectors and delivery channels to better understand the mechanisms and
strategies that determine how global development financing gets
distributed in developing countries. On this front, this study
problematizes the conventional wisdom on aid delivery tactics and donor
motives. While some have argued that government-to-government transfers
are primarily linked to non-development foreign policy goals for donors
while transfers that bypass the recipient government are primarily
linked to the promotion of public goods (Dietrich 2013; Steinwand 2015),
this study demonstrates that bypass aid is in fact tied to
non-development aid-for-policy exchanges (e.g., the buying and selling
of alliance commitments). This finding does not negate the view that
bypass aid is primarily directed toward the promotion of public goods.
To the contrary, and even more surprisingly, it shows that
non-development foreign policy goals can be a motivation for directly
supporting the promotion of public goods in developing countries. As
others have shown, the line between altruism and self-interest is blurry
(Bermeo 2017, 2018; Heinrich 2013).

\end{document}
