% Options for packages loaded elsewhere
\PassOptionsToPackage{unicode}{hyperref}
\PassOptionsToPackage{hyphens}{url}
%
\documentclass[
  11pt,
]{article}
\usepackage{lmodern}
\usepackage{amssymb,amsmath}
\usepackage{ifxetex,ifluatex}
\ifnum 0\ifxetex 1\fi\ifluatex 1\fi=0 % if pdftex
  \usepackage[T1]{fontenc}
  \usepackage[utf8]{inputenc}
  \usepackage{textcomp} % provide euro and other symbols
\else % if luatex or xetex
  \usepackage{unicode-math}
  \defaultfontfeatures{Scale=MatchLowercase}
  \defaultfontfeatures[\rmfamily]{Ligatures=TeX,Scale=1}
  \setmainfont[]{cochineal}
  \setmonofont[]{Fira Code}
\fi
% Use upquote if available, for straight quotes in verbatim environments
\IfFileExists{upquote.sty}{\usepackage{upquote}}{}
\IfFileExists{microtype.sty}{% use microtype if available
  \usepackage[]{microtype}
  \UseMicrotypeSet[protrusion]{basicmath} % disable protrusion for tt fonts
}{}
\makeatletter
\@ifundefined{KOMAClassName}{% if non-KOMA class
  \IfFileExists{parskip.sty}{%
    \usepackage{parskip}
  }{% else
    \setlength{\parindent}{0pt}
    \setlength{\parskip}{6pt plus 2pt minus 1pt}
    }
}{% if KOMA class
  \KOMAoptions{parskip=half}}
\makeatother
\usepackage{xcolor}
\IfFileExists{xurl.sty}{\usepackage{xurl}}{} % add URL line breaks if available
\urlstyle{same} % disable monospaced font for URLs
\usepackage[margin=1in]{geometry}
\setlength{\emergencystretch}{3em} % prevent overfull lines
\providecommand{\tightlist}{%
  \setlength{\itemsep}{0pt}\setlength{\parskip}{0pt}}
\setcounter{secnumdepth}{-\maxdimen} % remove section numbering

\ifluatex
  \usepackage{selnolig}  % disable illegal ligatures
\fi

\author{}
\date{09 November 2022}

% Jesus, okay, everything above this comment is default Pandoc LaTeX template. -----
% ----------------------------------------------------------------------------------
% I think I had assumed beamer and LaTex were somehow different templates.


\usepackage{kantlipsum}

\usepackage{abstract}
\renewcommand{\abstractname}{}    % clear the title
\renewcommand{\absnamepos}{empty} % originally center

\renewenvironment{abstract}
 {{%
    \setlength{\leftmargin}{0mm}
    \setlength{\rightmargin}{\leftmargin}%
  }%
  \relax}
 {\endlist}

\makeatletter
\def\@maketitle{%
  \newpage
%  \null
%  \vskip 2em%
%  \begin{center}%
  \let \footnote \thanks
      {\fontsize{18}{20}\selectfont\raggedright  \setlength{\parindent}{0pt} \@title \par}
    }
%\fi
\makeatother


 



%\author{}


\date{}

\usepackage{titlesec}

% 
\titleformat*{\section}{\large\bfseries}
\titleformat*{\subsection}{\normalsize\itshape} % \small\uppercase
\titleformat*{\subsubsection}{\normalsize\itshape}
\titleformat*{\paragraph}{\normalsize\itshape}
\titleformat*{\subparagraph}{\normalsize\itshape}

% add some other packages ----------

% \usepackage{multicol}
% This should regulate where figures float
% See: https://tex.stackexchange.com/questions/2275/keeping-tables-figures-close-to-where-they-are-mentioned
\usepackage[section]{placeins}



\makeatletter
\@ifpackageloaded{hyperref}{}{%
\ifxetex
  \PassOptionsToPackage{hyphens}{url}\usepackage[setpagesize=false, % page size defined by xetex
              unicode=false, % unicode breaks when used with xetex
              xetex]{hyperref}
\else
  \PassOptionsToPackage{hyphens}{url}\usepackage[draft,unicode=true]{hyperref}
\fi
}

\@ifpackageloaded{color}{
    \PassOptionsToPackage{usenames,dvipsnames}{color}
}{%
    \usepackage[usenames,dvipsnames]{color}
}
\makeatother
\hypersetup{breaklinks=true,
            bookmarks=true,
            pdfauthor={},
             pdfkeywords = {},  
            pdftitle={},
            colorlinks=true,
            citecolor=blue,
            urlcolor=blue,
            linkcolor=magenta,
            pdfborder={0 0 0}}
\urlstyle{same}  % don't use monospace font for urls

% Add an option for endnotes. -----



% This will better treat References as a section when using natbib
% https://tex.stackexchange.com/questions/49962/bibliography-title-fontsize-problem-with-bibtex-and-the-natbib-package



% set default figure placement to htbp
\makeatletter
\def\fps@figure{htbp}
\makeatother



\linespread{1.05}

\newtheorem{hypothesis}{Hypothesis}



\newcommand{\blankline}{\quad\pagebreak[2]}
\usepackage{graphicx}

\begin{document}



\hfill
\begin{minipage}[t]{1\textwidth}
\raggedleft%
{\bfseries  }\\[.35ex]
\emph{\small Denison University\\
100 W College St,\\
Granville, OH} \\[.35ex]

 \small{\tt \href{mailto:williamsmd@denison.edu}{\nolinkurl{williamsmd@denison.edu}}} \\ 
 \small{\href{http://true}{\tt true}}\\ 
\hspace{1cm} \\
 09 November 2022 \\ 
\end{minipage}

% \vspace*{1em} 

\vspace*{1em}

Dear Editorial Team:

\vspace*{1em}

 

% 09 November 2022
% 
% \vspace*{1em} 
% 

% % \setlength{\parindent}{16pt}
% \setlength{\parskip}{0pt}
% 
I would like to submit this original research article entitled ``Aid for
Alliances? Buying and Selling Alliances in the Political Economy of
Aid'' for consideration by \emph{International Studies Quarterly}. This
manuscript is original, has not been published previously, and is not
currently under review or being considered elsewhere.

In this paper, I show that developed countries target more aid in
developing countries with which they share a nonaggression pact, and
less aid in developing countries with which they share a defensive pact.
This finding is significant in light of the conventional view shared
among IR scholars that donor countries reward allies with greater aid. I
show, instead, that the nature of alliance commitments makes a
difference in whether donors reward or (surprisingly) cut aid to allies.
This is a distinction that past research has failed to address and its
recognition is important both for understanding the political objectives
of foreign aid donors and the substantive impact of the content of
alliance promises on signatory behavior.

I propose a logic of aid-for-alliance exchange where the substance and
direction of alliance promises is relevant for understanding why donors
might increase or cut foreign aid in allies. Of the two most prevalent
kinds of alliances that developed and developing countries
sign---nonaggression pacts and defensive pacts---the first has been
shown by previous studies to correspond with efforts to foster
cooperation in the wake of heightened tension and mistrust. The second,
meanwhile, reflects a commitment to militarily intervene if an ally is
attacked. I posit that the first kind of alliance reflects a kind of
aid-for-policy exchange, where donors use aid as a material incentive to
get a recipient to agree to a nonaggression pact. The second kind of
alliance, alternatively, reflects an asymmetric security commitment from
developed country donors. Due to the differences in material
capabilities between developed and developing countries, the offer of
defense in the face of attack disproportionately benefits developing
country signatories rather than the generally more powerful developed
countries. My findings are consistent with this logic and even endure
when the mode of aid delivery is taken into account.

Given the ubiquity of the view that donors reward allies with greater
aid, these findings should be of interest to readers of
\emph{International Studies Quarterly}. Research on alliances and on
foreign aid have made regular appearances in this journal, but rarely
are they examined together. For this reason, the simplistic idea that
allies receive more foreign aid has endured. This study puts this
conventional wisdom to the test and finds it wanting.

I have no conflicts of interest to disclose. All correspondence
concerning this manuscript can be addressed to me at
\texttt{williamsmd@denison.edu}.

Thank you for your consideration.

Best regards,

Dr.~Miles D. Williams, Ph.D.

Visiting Assistant Professor, Data for Political Research

Denison University

\end{document}
